% !TeX root = ../main.tex

\chapter{Introduction}\label{chapter:introduction}

\section{Why did we build this system?}
Why did we build this system?\\ 
Why did we need it?\\
The answer is actually rather straight forward:\\
Because we tend to forget about our convictions when we are hungry.\\
We all set dietary goals, but only a few of us follow through with them.\\
This is why we are impressed when we meet people who have followed through with their goals.\\
Because following through with your goals is actually hard.\\
But what if there was a system that could remind you of your goals just before you ate?
A system that could learn your dietary patterns and intevene when you need it to?\\
We have built this system as an answer to this question, and this thesis documents the building process.

\section{Goal and Structure of the Thesis}
The goal of this thesis is to explain the process of building and evaluating a system that learns a user's dietary patterns.
We first build a chatbot that enables users to log their eating patterns.
Next, we build a machine learning pipeline that takes a user's dietary data and uses it to train predictive models.
Using these models we predict when the user is going to eat again.
We start sending the user messages at the predicted times, and ask for feedback on the timing.
The system fine tunes the timing based on this feedback.

After building the system, we ask some volunteers to use it.
The volunteers use the system for some time and afterwards, we take a look at how the system performed.

In order to build this system, we first need to be familiar with some natural language processing and machine learning concepts.
We explain these concepts in chapter 2.
Afterwards, in chapter 3, we show how we build the system based on this information.
Next, at Chapter 4, we take a look at the data collected by our system and the models trained on this data.
At Chapter 5, we discuss the collected data, the user feedback on the chatbot and the trained models.
Finally, at Chapter 6, we mention some possible improvements on this work.

