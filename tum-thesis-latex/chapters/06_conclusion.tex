% !TeX root = ../main.tex

\chapter{Conclusion}\label{chapter:conclusion}
We have built a system that facilitates communication with the user and makes somewhat accurate predictions.
However, there is a lot of room for improvement.
In the next section, we will be going over some possible future work on this topic.
\section{Future Work}
\subsection{Solving the Initial Lack of Data}
One of the biggest problems concerning our system is the initial lack of data.
This lack of data prevents predictive functionality until the user logs enough entries.
However, there might be some possible solutions to this problem.

One possible solution would be to ask the users do describe their eating habits at registration.
This information could be used for initial values for time prediction, which could be optimized over time as the user logs more entries.

Another possibility is to use a clustering algorithm in order to categorize users' eating patterns.
During the registration phase, the new users would answer a series of questions about their eating patterns.
Users would be categorized based on their answers.
After the categorization, randomly sampled data from the other users in the same category could be used to train an initial predictive model for the user.

It is also possible to combine these two approaches by categorizing the users based on eating patterns, but still consulting the user about concrete meal times.

These systems could be evaluated by comparing how well the initial predictive model approximates the late stage predictive model trained on the actual user data.

\subsection{Reminders}
One missing component of out dialog agent that is much desired is the ability to send reminders.
While sending a reminder is easy, timing and frequency of reminders is not as trivial.
One possible approach would be to base the reminders on the predictions.
If the user doesn't log anything for a certain time after a prediction, a reminder could be sent.
But what 'certain time' is appropriate for this task?

What about the frequency?
While sending too few reminders is obviously not optimal, 
sending too many reminders could easily agitate the users.

These questions could be answered by asking the user for feedback on the reminders.
The user could rate the appropriateness of the reminder,
and the reminder component could be iteratively optimized based on these ratings.

The resulting component could be evaluated based on the number of reminders that received good ratings.

\subsection{Better Notification Timing}
We focus on sending users notifications when they are about to eat.
However, this is not the only time that people make food related choices.
Consider grocery shopping for example, which dictates what the user eats at home for the next days.

It could be possible to build a system that could notify the user whenever a food related choice is about to be made.
For example, the user could notify the system before certain events, such as shopping, cooking or going to a restaurant.
A predictive model could be trained that processes user location and time to determine if such an event is about to occur. 
Such a model could be evaluated based on recall; number of accurately predicted food related events divided by the total number of food related events.

\subsection{Detailed Meal Logging}
When logging a new entry, we only differentiate between meals and snacks.
One improvement over this would be to let the user log exactly what they eat.
The system would need to calculate the approximate nutrient intake for every entry,
and make recommendations on what to eat based on this information and the user goal.

